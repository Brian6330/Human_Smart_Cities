\documentclass[a4paper,12pt]{article}
\usepackage{unifrsr}

\begin{document}
\seminartitle{Human Smart Cities Seminar} % The title of the seminar

\title{
   Insert specific topic title \\
   \large Smart Government: Transparency and Open Data} % The title of your project


\author{
   Elias Wipfli (13-123-922) 
   \thanks{\email{elias.wipfli@students.unibe.ch}, University of Bern}
   \and
   Julius Oeftiger (16-127-532) 
   \thanks{\email{julius.oeftiger@students.unibe.ch}, University of Bern}
   \and
   Brian Schweigler (16-102-071)
   \thanks{\email{brian.schweigler@students.unibe.ch}, University of Bern}
}

\supervisor{Prof. Edy Portmann} % Name of the supervisor

\assistant{Minh Tue Nguyen \and Moreno Colombo \and Jhonny Pincay} %Name of the assistant(s)

\date{31 December 2021} % Note: if this is left out, today's date will be used, this is the submission date!

\maketitle

\begin{abstract}
This package, \textsf{unifrsr}, allows the easy creation of seminar
reports using standard \LaTeX. It should be the last
package loaded, to ensure that nothing overrides the page layout.

Note that utility commands are available: 
\begin{itemize}
\item \verb+\seminartitle+ which specifies, optionally, the title of the seminar the report is being written for
\item \verb+\supervisor+ which specifies the name of the professor supervisor of the project
\item \verb+\assistant+ which specifies, optionally, the name of the assistants of the seminar, separated by the command \verb+\and+
\end{itemize}

\keywords{Seminar report, Human-IST Research Institute}

\end{abstract}

\tableofcontents
\newpage

\section{Introduction}
This is an example of a section. As you see, it is just a standard
{\LaTeX} section. Here is a footnote%
\footnote{This is a footnote}.

\subsection{Subsection of introduction}
It's all standard \LaTeX, as you can see.
This is a very nice paper:~\cite{zadeh}.

\section{Human Smart Cities} % Chapter that relates the project to and talks about Human Smart Cities (re: Presentation feedback)

\section{Methodology} % What we are doing, how we are doing it

\section{Evaluation} % How we are evaluating our work.


\section {Result} % The results of our work, as well as the evaluation


\section{Conclusion} % Critical look at our work


\section{References}

\bibliography{references}
\bibliographystyle{plain}

\end{document}
