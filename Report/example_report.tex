\documentclass[a4paper,12pt]{article}
\usepackage{unifrsr}

\begin{document}
\seminartitle{Human Smart Cities Seminar} % The title of the seminar

\title{An example of the use of the seminar report package} % The title of your project

\author{Moreno Colombo (XX-XXX-XXX)\thanks{\email{moreno.colombo@unifr.ch}, University of Fribourg}
   %\and Jhon Smith (YY-YYY-YYY)\thanks{\email{john.smith@email.ch}, University of University}
   }	% Note: XX-XXX-XXX is the student's matriculation number
   % The author(s), separated by \and

\supervisor{Prof. Edy Portmann} % Name of the supervisor

\assistant{Moreno Colombo \and Jhonny Pincay} %Name of the assistant(s)

\date{26 September 2018} % Note: if this is left out, today's date will be used, this is the submission date!

\maketitle

\begin{abstract}
This package, \textsf{unifrsr}, allows the easy creation of seminar
reports using standard \LaTeX. It should be the last
package loaded, to ensure that nothing overrides the page layout.

Note that utility commands are available: 
\begin{itemize}
\item \verb+\seminartitle+ which specifies, optionally, the title of the seminar the report is being written for
\item \verb+\supervisor+ which specifies the name of the professor supervisor of the project
\item \verb+\assistant+ which specifies, optionally, the name of the assistants of the seminar, separated by the command \verb+\and+
\end{itemize}

\keywords{Seminar report, Human-IST Research Institute}

\end{abstract}

\tableofcontents
\newpage

\section{Introduction}
This is an example of a section. As you see, it is just a standard
{\LaTeX} section. Here is a footnote%
\footnote{This is a footnote}.

\subsection{Subsection of introduction}
It's all standard \LaTeX, as you can see.
This is a very nice paper:~\cite{zadeh}.

\bibliography{references}
\bibliographystyle{plain}

\end{document}
