\documentclass[a4paper,12pt]{article}
\usepackage{unifrsr}

\begin{document}
\seminartitle{Human Smart Cities Seminar} % The title of the seminar

\title{
   Insert specific topic title (TODO) \\ 
   \large Smart Government: Transparency and Open Data} % The title of your project


\author{
   Elias Wipfli (13-123-922) 
   \thanks{\email{elias.wipfli@students.unibe.ch}, University of Bern}
   \and
   Julius Oeftiger (16-127-532) 
   \thanks{\email{julius.oeftiger@students.unibe.ch}, University of Bern}
   \and
   Brian Schweigler (16-102-071)
   \thanks{\email{brian.schweigler@students.unibe.ch}, University of Bern}
}

\supervisor{Prof. Edy Portmann} % Name of the supervisor

\assistant{Minh Tue Nguyen \and Moreno Colombo \and Jhonny Pincay} %Name of the assistant(s)

\date{31 December 2021} % Note: if this is left out, today's date will be used, this is the submission date!

\maketitle

\begin{abstract}


\keywords{Seminar report, Human-IST Research Institute, Human Smart Cities, Smart Governance}

\end{abstract}

\tableofcontents
\newpage

\section{Introduction}
Cities are adopting Information and Communication Technologies (ICT) more and more into their services.
Thus, the need to address challenges in such Human Smart Cities keeps rising.
As the citizens are the main drivers of change, they must be included to best handle these challenges \cite{oliveira_smart_2015}.

One such challenge that must be addressed is the dispersion of information among different systems and applications.
Must importantly, this information must be found within a short time-frame.
If parts of the retrieved data is unclear, talking to an expert might be of importance.
This leads to the follow-up issue of determining who is likely to be an expert in a specific field.

The Research Questions focused on are:
\begin{enumerate}
\item Given multiple text files, how can their contents be searched to return the best fitting ones?  
\item How can a list of experts be extracted from the submitted text files?  
\item The submitting user of a given text file, must not necessarily be the user who wrote it. 
      How can this be considered when determining the expert list?
\end{enumerate}

The remainder of this project report is structured as follows: 
the required background to understand certain terminology and methods used is explained in chapter 2, 
the methodology used is thoroughly explained in chapter 3, 
in chapter 4 the evaluation procedure is described, 
in chapter 5 the project results will be presented 
and in chapter 6 the report will be concluded while also providing a short outlook on possible extensions to the work.

% It's all standard \LaTeX, as you can see.
% This is a very nice paper:~\cite{zadeh}.

\section{Theory} % Chapter that relates the project to and talks about Human Smart Cities (re: Presentation feedback)
% Human Smart Cities
How does SOLR work?
"Solr works by gathering, storing and indexing documents from different sources and making them searchable in near real-time. 
It follows a 3-step process that involves indexing, querying, and finally, ranking the results – all in near real-time, even though it can work with huge volumes of data." % https://sematext.com/guides/solr/

What is a schema:
Define a schema. The schema tells Solr about the contents of documents it will be indexing. 
In the online store example, the schema would define fields for the product name, description, price, manufacturer, and so on. 
Solr’s schema is powerful and flexible and allows you to tailor Solr’s behavior to your application. % https://solr.apache.org/guide/8_11/a-quick-overview.html  

Why SOLR?

What did we need to understand that readers may need to understand to?

How is the expert list built?


\section{Methodology} % What we are doing, how we are doing it
Coded in python.

SOLR Set-up?

How is GUI created?

How exactly is expert list determined?


\section{Evaluation} % How we are evaluating our work.
Precision and Recall?

EER?

How or rather what can we even evaluate?

% \subsection{Subsection}
% \footnote{This is a footnote}.

\section {Result} % The results of our work, as well as the evaluation
What happened?

What can we show as a result? (Screenshots? Walk-through of an example?)

What is new to our approach?

\section{Conclusion} % Critical look at our work
Summarizing our findings.

What could future projects improve on?

What can we improve upon?

What is the main take-away?

\section{References}

\bibliography{references}
\bibliographystyle{plain}

\end{document}
